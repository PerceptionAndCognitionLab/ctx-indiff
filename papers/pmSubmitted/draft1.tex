\documentclass[american,man]{apa6}

\usepackage{amssymb,amsmath}
\usepackage{ifxetex,ifluatex}
\usepackage{fixltx2e} % provides \textsubscript
\ifnum 0\ifxetex 1\fi\ifluatex 1\fi=0 % if pdftex
  \usepackage[T1]{fontenc}
  \usepackage[utf8]{inputenc}
\else % if luatex or xelatex
  \ifxetex
    \usepackage{mathspec}
    \usepackage{xltxtra,xunicode}
  \else
    \usepackage{fontspec}
  \fi
  \defaultfontfeatures{Mapping=tex-text,Scale=MatchLowercase}
  \newcommand{\euro}{€}
\fi
% use upquote if available, for straight quotes in verbatim environments
\IfFileExists{upquote.sty}{\usepackage{upquote}}{}
% use microtype if available
\IfFileExists{microtype.sty}{\usepackage{microtype}}{}

% Table formatting
\usepackage{longtable, booktabs}
\usepackage{lscape}
% \usepackage[counterclockwise]{rotating}   % Landscape page setup for large tables
\usepackage{multirow}		% Table styling
\usepackage{tabularx}		% Control Column width
\usepackage[flushleft]{threeparttable}	% Allows for three part tables with a specified notes section
\usepackage{threeparttablex}            % Lets threeparttable work with longtable

% Create new environments so endfloat can handle them
\newenvironment{ltable}
  {\begin{landscape}\begin{center}\begin{threeparttable}}
  {\end{threeparttable}\end{center}\end{landscape}}

\newenvironment{lltable}
  {\begin{landscape}\begin{center}\begin{ThreePartTable}}
  {\end{ThreePartTable}\end{center}\end{landscape}}

\usepackage{ifthen} % Only add declarations when endfloat package is loaded
\ifthenelse{\equal{man}{\string jou}}{%
  \DeclareDelayedFloatFlavor{ThreePartTable}{table} % Make endfloat play with longtable
  \DeclareDelayedFloatFlavor{ltable}{table} % Make endfloat play with lscape
  \DeclareDelayedFloatFlavor{lltable}{table} % Make endfloat play with lscape & longtable
}{}%


% The following enables adjusting longtable caption width to table width
% Solution found at http://golatex.de/longtable-mit-caption-so-breit-wie-die-tabelle-t15767.html
\makeatletter
\newcommand\LastLTentrywidth{1em}
\newlength\longtablewidth
\setlength{\longtablewidth}{1in}
\newcommand\getlongtablewidth{%
 \begingroup
  \ifcsname LT@\roman{LT@tables}\endcsname
  \global\longtablewidth=0pt
  \renewcommand\LT@entry[2]{\global\advance\longtablewidth by ##2\relax\gdef\LastLTentrywidth{##2}}%
  \@nameuse{LT@\roman{LT@tables}}%
  \fi
\endgroup}


\ifxetex
  \usepackage[setpagesize=false, % page size defined by xetex
              unicode=false, % unicode breaks when used with xetex
              xetex]{hyperref}
\else
  \usepackage[unicode=true]{hyperref}
\fi
\hypersetup{breaklinks=true,
            pdfauthor={},
            pdftitle={Developing Constraint in Bayesian Mixed Models},
            colorlinks=true,
            citecolor=blue,
            urlcolor=blue,
            linkcolor=black,
            pdfborder={0 0 0}}
\urlstyle{same}  % don't use monospace font for urls

\setlength{\parindent}{0pt}
%\setlength{\parskip}{0pt plus 0pt minus 0pt}

\setlength{\emergencystretch}{3em}  % prevent overfull lines

\setcounter{secnumdepth}{0}
\ifxetex
  \usepackage{polyglossia}
  \setmainlanguage{}
\else
  \usepackage[american]{babel}
\fi

% Manuscript styling
\captionsetup{font=singlespacing,justification=justified}
\usepackage{csquotes}
\usepackage{upgreek}



\usepackage{tikz} % Variable definition to generate author note

% fix for \tightlist problem in pandoc 1.14
\providecommand{\tightlist}{%
  \setlength{\itemsep}{0pt}\setlength{\parskip}{0pt}}

% Essential manuscript parts
  \title{Developing Constraint in Bayesian Mixed Models}

  \shorttitle{Developing Constraint in Bayesian Mixed Models}


  \author{
          Julia M. Haaf\textsuperscript{1},
          Jeffrey N. Rouder\textsuperscript{1}  }

  \def\affdep{{"", ""}}%
  \def\affcity{{"", ""}}%

  \affiliation{
    \vspace{0.5cm}
          \textsuperscript{1} University of Missouri  }


%   \def\affinst{{"init", "University of Missouri"}}%
%   \def\affstate{{"init", ""}}%
%   \def\affcntry{{"init", ""}}%

 % If no note is defined give only author information if available
    \note{
    \vspace{1cm}
    Author note

    \raggedright
    \setlength{\parindent}{0.4in}

    \newcounter{author}

%     %       %       \setcounter{author}{0}
%         %           \addtocounter{author}{1}
%         %         \expandafter\edef\csname authorid\endcsname{\theauthor}
%         Julia M. Haaf, \pgfmathparse{\affdep[\authorid]} \pgfmathresult, \pgfmathparse{\affinst[\authorid]} \pgfmathresult, \pgfmathparse{\affcity[\authorid]} \pgfmathresult, \pgfmathparse{\affstate[\authorid]} \pgfmathresult, \pgfmathparse{\affcntry[\authorid]} \pgfmathresult
%       %     ;
%     %       %       \setcounter{author}{0}
%         %           \addtocounter{author}{1}
%         %         \expandafter\edef\csname authorid\endcsname{\theauthor}
%         Jeffrey N. Rouder, \pgfmathparse{\affdep[\authorid]} \pgfmathresult, \pgfmathparse{\affinst[\authorid]} \pgfmathresult, \pgfmathparse{\affcity[\authorid]} \pgfmathresult, \pgfmathparse{\affstate[\authorid]} \pgfmathresult, \pgfmathparse{\affcntry[\authorid]} \pgfmathresult
%       %     .

                      Correspondence concerning this article should be addressed to Julia M. Haaf, 205 McAlester Hall, University of Missouri, Columbia, MO 65211. E-mail: \href{mailto:jhaaf@mail.missouri.edu}{\nolinkurl{jhaaf@mail.missouri.edu}}
                          }
  
  \abstract{Model comparison in Bayesian mixed models is becoming popular in
psychological science. Here we develop a set of nested models that
account for order restrictions across individuals in psychological
tasks. An order-restricted model addresses the question \enquote{Does
Everybody}, as in, \enquote{Does everybody show the usual Stroop
effect}, or \enquote{Does everybody respond more quickly to intense
noises than subtle ones.} The crux of the modeling is the instantiation
of 10s or 100s of order restrictions simultaneously, one for each
participant. To our knowledge, the problem is intractable in frequentist
contexts but relatively straightforward in Bayesian ones. We develop a
Bayes-factor model-comparison strategy using Zellner and colleagues'
default \(g\)-priors appropriate for assessing whether effects obey
equality and order restrictions. We apply the methodology to seven data
sets from Stroop, Simon, and Eriksen interference tasks. Not too
surprisingly, we find that everybody Stroops---that is, for all people
congruent colors are truly named more quickly than incongruent ones.
But, perhaps surprisingly, we find these order constraints are violated
for some people in the Simon task, that is, for these people spatially
incongruent responses occur truly more quickly than congruent ones!
Implications of the modeling and conjectures about the differences in
task dynamics are discussed.}
  \keywords{Bayesian mixed models, Bayes factors, Individual differences, Order
constraints, Equality constraints, Priming \\

    
  }

  \usepackage{bm}
  \usepackage{pcl}
  \usepackage{amsmath}
  \usepackage{setspace}

\begin{document}

\maketitle



\section{Introduction}\label{introduction}

\begin{figure}

{\centering \includegraphics{draft1_files/figure-latex/fig_1-1} 

}

\caption{Hypothetical distribution of individuals' true effects (solid lines) and the resulting model on the average effect (dashed lines).  Panel A shows the usual case where individuals' effects are graded and there is no special role for the direction of the effect.  Panel B shows the ''Everone Stroops'' case where individuals' effects are constrained to be positive.  The resulting distribution for the average effect is the same for both cases}\label{fig:fig_1}
\end{figure}

\begin{figure}

{\centering \includegraphics{draft1_files/figure-latex/emp-1} 

}

\caption{Observed difference scores, $\delta_i$, for 121 individuals. The scores are ordered from smallest to largest, and the open circles denote negatively valued differences.  The dashed horizontal line is the mean difference, 65 ms, and the error bars are 95\% confidence intervals for each individual.}\label{fig:emp}
\end{figure}

Many experimental tasks in psychology have a massively repeated
character in which participants perform many trials in a small number of
conditions. For example, in most social-cognitive priming tasks,
participants perform hundreds of trials in the primed and unprimed
conditions (Amodio et al., 2004). In reading tasks, participants may
read hundreds of words (e.g. Balota, Cortese, Sergent-Marshall, Spieler,
\& Yap, 2004); in perceptual tasks, participants may identify hundreds
of items (e.g. Swagman, Province, \& Rouder, 2015); in memory tasks,
participants may be asked to recognize hundreds of previously-studied
memoranda. Examples in decision making and attention spring immediately
to mind.

One of the common questions researchers ask in these tasks is whether
there is an effect of a manipulation on an outcome variable. For
example, in a priming task, a researcher may ask if responses when
primed are faster than those when not primed. Likewise, massively
repeated tasks may be used to ask whether the strength of a stimulus,
the context in which it is presented, or the attention directed toward
it affect the response.

If we limit our attention to tasks with two conditions, say as in the
priming case, the usual course is to aggregate across the repetitions to
produce a participant-by-condition mean score. These mean scores may be
subtracted across the conditions to produce a participant-specific
observed effects, and these effects may be tested with a \(t\)-test. If
the t-test null is rejected, then the researcher may conclude that there
is evidence for an average effect across the population of participants.

The great value of this aggregate approach is that it is simple and can
be performed by all researchers, including undergraduate and graduate
students. The limitation, however, is that researchers are unable to
address individual variation. A significant \(t\)-test does not
guarantee that all individuals display a priming effect. Some truly may
while others truly may not. Even more alarming, while some may truly
have the usual priming effect, others may truly have the opposite,
negative effect. Figure 1 shows two scenarios that yield identical
t-tests. In the first scenario, true individual priming effects follow a
normal, and there is sufficient variability that there are negative true
effects for a minority of participants. In the second scenario, all
participants are constrained to have a positive effect.

We think the differences between these scenarios are theoretically
important. We distinguish between two cases: In the first case all
participants have the same direction of their true effect (Figure 1B).
In this case, the mean effect is interpretable as a proxy for what all
participants do. This commonality of direction may be used to specify
common mechanisms, say that priming is automatic and beyond strategic
control (Greenwald, Klinger, \& Schuh, 1995), or that stimulus strength
has a simple, common neurological correlate, say that more strength
corresponds to greater neural firing rates in certain cell assemblies
(Roitman \& Shadlen, 2002).

In the second case some participants have a true effect in the usual
direction while others have an effect in the opposite direction. Figure
1A illustrates this case. A real-world example is handedness. Imagine a
task where people throw balls with their right and left hands, and we
ask for which hand did the ball go further. Here, right-handed people
are almost always stronger with their right hand; left-handed people
tend to be stronger with their left hand. In the population, more
individuals are right-handed. So on average, people may be stronger with
their right hand. But this average strength does not convey any
meaningfull information about all individuals in the sample. Thus, this
case corresponds to more complex theoretical implications that assume
more than one underlying mechanism. Indeed, handedness is complicated,
and it is better described as a syndrome than a single phenomenon(Coren,
1993). Other possible examples of complicated phenomena from the
literature are different routes of attitude formation (Sweldens,
Corneille, \& Yzerbyt, 2014) or different stategies in decision theory
(Kahneman \& Tversky, 1972).

Assessing whether true effects are all in one direction may seem simple,
but it is quite hard. For an illustration of this difficulty, consider
the following Stroop interference experiment from Von Bastian, Souza, \&
Gade (2015). In the classic Stroop interference task (Stroop, 1935),
participants are asked to name the color of displayed words. The words
themsevles are color names, such as RED and GREEN, and the meaning of
the word may be congruent with the color, i.e., RED displayed in red, or
incongruent with the color, i.e., GREEN displayed in red. Stroop
interference refers to the slowdown of color identification when the
meaning is incongruent, and it is a robust phenomenon (MacLeod, 1991).
The question at hand is not the existence of Stroop interference, but
whether all people display the effect in the usual direction.

Figure 2 shows the ordered Stroop interference effects, \(\delta_i\),
for 121 individuals from Von Bastian et al. (2015). People vary
considerably in the size of their effects: they range from -100 ms to
181 ms. The problem is that it is difficult to distinguish between noise
variation and true variation from the observed values. Furthermore, the
95\% confidence intervals around invidivduals' sample effects do not
provide any direct way of assessing whether all effects are positive. We
note that no CI in the figure is located exclusively below zero. Yet,
CIs are about each individual in isolation and cannot be used to answer
questions about a group of individuals. Even though no specific
individual is identified as definitively negative, there may be enough
evidence across the collection of individuals to state that some of
these individuals have true negative effects without identifying which
one.

The problem at hand is known as order-restricted inference, and there is
a voluminous frequentist literature on it (see Robertson, Wright, \&
Dykstra, 1988). Order-restricted inference is straightforward for
one-dimensional cases, say whether the grand mean is greater than zero
or not. The usual frequentist solution is to adjust rejection regions as
is done in one-sided \(t\)-tests. But the question here is whether a
separate order-restriction holds simultaneously for all individuals.
Therefore, we must assess 121 order restrictions simultaneously, and
this problem is much more complicated. Of note, usual AIC and BIC
approaches are not applicable because penalties reflect the number of
parameters rather than their direction (Klugkist \& Hoijtink, 2007).

Our approach to order-restricted inference is to analyze Bayesian mixed
models with order and equality constraints.\footnote{All analyses were
  conducted using R (3.3.1, R Core Team, 2016) and the R-packages
  \emph{abind} (1.4.3, Plate \& Heiberger, 2015), \emph{BayesFactor}
  (0.9.12.2, Morey \& Rouder, 2015), \emph{coda} (0.18.1, Plummer, Best,
  Cowles, \& Vines, 2006), \emph{curl} (0.9.7, Ooms, 2016),
  \emph{devtools} (1.11.1, Wickham \& Chang, 2016), \emph{fields}
  (8.4.1, Douglas Nychka, Reinhard Furrer, John Paige, \& Stephan Sain,
  2015), \emph{gmm} (1.5.2, Chauss{é}, 2010), \emph{maps} (3.1.1,
  Richard A. Becker, Ray Brownrigg. Enhancements by Thomas P Minka, \&
  Deckmyn., 2016), \emph{MASS} (7.3.45, Venables \& Ripley, 2002),
  \emph{Matrix} (1.2.6, Bates \& Maechler, 2016), \emph{MCMCpack}
  (1.3.6, Martin, Quinn, \& Park, 2011), \emph{msm} (1.6.1, Jackson,
  2011), \emph{mvtnorm} (1.0.5, Genz \& Bretz, 2009; Wilhelm \& G,
  2015), \emph{papaja} (0.1.0.9074, Aust \& Barth, 2015), \emph{plotrix}
  (3.6.1, Lemon, 2006), \emph{sandwich} (2.3.4, Zeileis, 2004, 2006),
  \emph{spam} (1.3.0, Furrer \& Sain, 2010; Gerber \& Furrer, 2015),
  \emph{spatialfil} (0.15, Dinapoli \& Gatta, 2015), and \emph{tmvtnorm}
  (1.4.10, Wilhelm \& G, 2015).} There are two advantages of Bayesian
methods in this context: 1. It is tractable. Bayesian analysis is
conceptually straightforward and computationally feasible. 2. Bayesian
methods, particularly Bayes factors, offer a calibration for inference
that is situated for order constraints (see also Klugkist \& Hoijtink,
2007).

Before describing the models we developed to assess constraints like the
one in right panel of Figure 1, it is worthwhile to ask if these
constraints are useful. We consider two critiques. The first is that the
order constraints are so natural and obvious that they assuredly hold
\emph{a priori}. For example, it is hard to imagine that anyone
identifies dim flashes of light more quickly than bright flashes, that
anyone reads long novel nonwords faster than short novel nonwords, or
that anyone forgets repeated items at a greater rate than unrepeated
items. Yet, we can think of examples where they do not hold such as the
handedness example provided previously. A second critique is the
diametric opposite of the above critique. It is that these order
constraints can never hold exactly. There must be somebody, somewhere
who has a negative true Stroop effect. This critique reminds us of the
one that the null is never true (Cohen, 1994), for which there are
several salient rebuttals (Rouder \& Morey, 2012). We consider
statements like \enquote{Everyone Stroops} to be of high value even if
they hold approximately or platonically. If broadly applicable in common
settings, they serve as important statements of constraint on theory. In
summary, we consider the two extreme positions---that order constraints
always hold or that order constraint never hold---to be intellectually
unsatisfying. The best course then is to assess the evidence from data
for these constraints.

\section{Models of Individual
Variation}\label{models-of-individual-variation}

\begin{figure}

{\centering \includegraphics{draft1_files/figure-latex/modelbytwo-1} 

}

\caption{{\footnotesize  Model specification and resulting predictions for two participants.  Panel $\mbox{A}_1$ shows the specification for the unstructured model $\calM_u$ for fixed $\nu$ and $\eta$.  Without structure, the bivariate density is that of a normal.  Panel $\mbox{B}_1$ shows the specification for the positive-effects model, and the bivariate density is restricted to the upper-right quandrant where effects are positive for both participants.  Panel $\mbox{C}_1$ shows the specification for the common-effect model; when the two participants are constrained to have the same effect, the resulting density is a line on the bivariate space.  Panel $\mbox{D}_1$ shows the specification for the null model, the point denotes that both participants have zero effect.  The right column, Panels $\mbox{A}_2$ - $\mbox{D}_2$, show the bivariate predictions for the respective models.  Correlations in panels $\mbox{A}_2$ and $\mbox{B}_2$} result from prior variation in the overall mean for the hierarchical specification.}\label{fig:modelbytwo}
\end{figure}

We explicitly consider experiments with two conditions, which are
generically termed \emph{control} and \emph{treatment} here. To model
individual variability in massively repeated designs, we adopt the
following notation: Let \(Y_{ijk}\) be a response variable, say response
time (RT), for the \(k\)th replicate for the \(i\)th participant,
\(i=1,\ldots,I\) in the \(j\)th condition, \(j=1,2\) with
\(k=1,\ldots,K_{ij}\). We place a linear model on \(Y_{ijk}\):

\begin{equation} \label{base1}
Y_{ijk} \stackrel{iid}{\sim} \mbox{Normal}\left([\mu+\alpha_i]+x_j[\theta_i],\sigma^2\right).
\end{equation}

Here, the term \((\mu+\alpha_i)\) serve as intercepts with \(\mu\) being
the grand mean of intercepts and \(\alpha_i\) being individual
deviations. The term \(x_j\) codes the condition, with \(x_j=0\) for the
control condition and \(x_j=1\) for the treatment condition. The effect
is the slope, \(\theta_i\), and the collection of these parameters
across participants are the target of interest.

We develop a series of four mixed models on these effect parameters. At
one end of the spectrum, the most general model with the least
constraint simply posits that the individual effects follow a graded
distribution. At the other end, the most constrained model specifies
that there is no effect for every individual. The following models are
ordered from the least constrained to the most, and the full set
provides a useful tool for assessing the constraint in effects. We then
use Bayesian model comparison, discussed subsequently, to assess what
level of constraint is most appropriate for a set of data.

\subsection{The Unstructured Model}\label{the-unstructured-model}

The unstructered model is denoted \(\calM_u\) and places no order or
equality constraints on the collection of effects:

\[
  \begin{array}{llr}
\calM_u: && \theta_i \stackrel{iid}{\sim} \mbox{Normal}(\nu,\eta^2),\\
\end{array}
\]

where \(\nu\) and \(\eta^2\) are population-level parameters describing
the mean and variance of effects across the population. This model is
quite flexible in that all combinations of individual effects are
plausible. Figure 3\(\mbox{A}_1\) shows this flexiblity for two
participants. The true-effect value for the first individual is shown on
the \(x\)-axis, the true-effect value for the second is shown on the
\(y\)-axis. As can be seen, combinations of effects in the same and
opposite directions for the two individuals have plausibility. Here, for
illustration, \(\nu\) is set to 0 ms and \(\eta\) is set to 230 ms. In
application, \(\nu\) and \(\eta^2\) are treated as parameters that are
free to vary as a function of data.

\subsection{The Positive-Effects
Model}\label{the-positive-effects-model}

The positive-effects model, denoted \(\calM_+\), captures the constraint
that each individual has a true effect in the predicted direction. \[
  \begin{array}{llr}
\calM_+: & & \theta_i \stackrel{iid}{\sim} \mbox{Normal}_+(\nu,\eta^2),\\
  \end{array}
\] where \(\mbox{Normal}_+\) denotes a truncated normal distribution
with a lower bound at zero. The probability density distribution of
\(\theta_i\) for two participants is shown in Figure 3\(\mbox{B}_1\).
Note that the distribution in the figure is darker, that is more
\emph{dense}, than the distribution in Figure 3\(\mbox{A}_1\). The
reason for that is the reduction of space. The space of valid parameter
values in this illustration is \(\frac{1}{4}\)th the space of valid
parameter values in the unstructured model as the positive-effect model
occupies only 1 of the 4 quandrants. This reduction of space relative to
the unstructued model becomes larger as additional participants are
considered.

\subsection{The Common-Effect Model}\label{the-common-effect-model}

The common-effect model, denoted \(\calM_1\), captures the constraint
that each individual has the same effect:\\\[
  \begin{array}{llrr}
\calM_1: & & \theta_i = \nu.\\
  \end{array}
\]

This common-effect model seems \emph{a priori} unlikely as it posits a
constant effect, for example a common priming effect, across all
individuals. Yet, we include it here for two reasons: First, it is a
logical null that can be used to benchmark claims of individual
differences. If this common-effects model provides a superior
description and yet we view it as unlikely, we may then turn our
attention to the adequacy of the design to capture individual
differences. In this regard, the model serves as a valuable design
check. Second, we think commonalities should be given more \emph{a
priori} weight. Important invariances, even if they hold approximately,
serve as good structure to build theory. Hence, if individual effect
vary to a very small extend, \(\calM_1\) could still be the preferred
model.

For this model, \(\calM_1\), the distribution of \(\theta_i\) for two
participants follows the diagonal as shown in Figure 3\(\mbox{C}_1\).
The geometry of the constraint is reducing a volume to a single line.

\subsection{The Null Model}\label{the-null-model}

The null model, denoted \(\calM_0\), specifies that each participant's
true effect is identically zero \[
  \begin{array}{llr}
\calM_0: & & \theta_i = 0.\\
  \end{array}
\] As everyone's effect is fixed to zero in this model, the distribution
of \(\theta_i\) for two participants shown in Figure 3\(\mbox{D}_1\) is
simply a point at zero.

\subsection{Additional Specifications}\label{additional-specifications}

We analyze the above four models in a Bayesian framework. Additional
specifications are needed for parameters \(\mu\), \(\sigma\), \(\nu\),
\(\eta^2\), and the collection of \(\alpha_i\). Our choices are
motivated by the following two considerations: First, we adopt
specifications that lead to computationally convenient algorithms for
model comparison. Second, we adopt specifications that adhere to a
subjective Bayesian philosophy (DeGroot, 1982; Goldstein, 2006; Rouder,
Morey, \& Wagenmakers, 2016), where priors are weakly informative and
reflect a reasonable range of beliefs. Here, the prior settings reflect
our \emph{a priori} general substantive knowledge of generic data
observed in these tasks.

We adopt what is known as a Zellner \(g\)-prior specification (Zellner,
1986). This specification is well studied and has been influential in
linear modeling in statistics (e.g. Bayarri \& Garcia-Donato, 2007; F.
Liang, Paulo, Molina, Clyde, \& Berger, 2008; Overstall \& Forster,
2010). It underlies most Bayes factor development in psychology
including development by Rouder, Morey, Wagenmakers and colleagues
(Rouder:etal:2012; Rouder \& Morey, 2012; Rouder, Speckman, Sun, Morey,
\& Iverson, 2009; R. Wetzels \& Wagenmakers, 2012). In this context, we
follow the ANOVA development by Rouder, Morey, Speckman, \& Province
(2012) as follows:

Consider the following \(g\)-prior specification on \(\alpha_i\):

\begin{equation} \label{alpha}
\alpha_i | g_{\alpha},\sigma^2 \stackrel{iid}{\sim} \mbox{Normal}(0, \sigma^2 g_{\alpha}).
\end{equation}

In the g-prior specification the prior on effect parameters, in this
case \(\alpha_i\), is a function of \(\sigma^2\). The role of
\(\sigma^2\) in the prior may not be transparent. It is helpful to
present an equivalent specificiation in \emph{standardized effect sizes}
rather than in effects. Let \(\alpha_i^*\) be the \(i\)th individual's
intercept effect size, where \(\alpha_i^*=\alpha_i/\sigma\). The model
reparameterized in this effect-size parameter may be given as
\(Y_{ijk} \sim \mbox{Normal}([\mu+\sigma\alpha_i^*]+x_j\theta_i,\sigma^2)\),
where the prior on \(\alpha_i^* \sim \mbox{Normal}(0, g_{\alpha})\).
Here we see there is no mystery placing the \(\sigma^2\) in the prior
variance in (\ref{alpha}). It allows us to focus on \(g\), which is the
prior variance on effect sizes. Effect sizes are convenient because
psychologists have developed extensive intuition about how these should
vary across manipulations and domains.

In \(g\)-prior specifications, priors may be placed on the \(g\)
parameter. This additional specification is less assumptive and
represents more uncertainty for the analyst. The usual specification
(Zellner \& Siow, 1980), which we follow, is a scaled inverse-\(\chi^2\)
with one degree-of-freedom.\footnote{The density of the inverse
  \(\chi^2\) distribution is
  \(f(\sigma^2;b)=\frac{\sqrt{b}}{\Gamma(1/2)}\left(\sigma^2\right)^{-3/2}\mbox{exp}\left(\frac{-b}{\sigma^2}\right)\),
  where \(b\) is a scale parameter.} \[
g_{\alpha} \sim \mbox{Inverse-$\chi^2$} (r_{\alpha}^2).
\] The quantity \(r_{\alpha}^2\) is a scaling setting on variability (in
standardized units), and it must be set \emph{a priori}.

Figure 4A helps provide guidance for setting \(r_\alpha\). It shows the
marginal prior for two different individuals' \(\alpha\) parameters. The
distribution is centered at zero (insuring that \(\mu\) is the grand
mean). The value of \(r_\alpha\) sets the scale, and the plot shows the
case for \(r_\alpha=1\). We show the values in time units (ms) and in
effect-size units relative to a value of \(\sigma = 300\) ms. We can use
our substantive knowledge of response times to set \(r_\alpha\). In the
analyzed tasks, response times are measured for the identification of
simple stimulus properties such as the color of a word. Response times
in simple choice tasks tend to have standard deviations around 200 ms to
400 ms for repeated trials within a person (Luce, 1986). The average,
300 ms, serves as a ballpark expectation for \(\sigma\). People also
tend to vary from one another by about the same amount of 300 ms. For
example, one person's mean might be 500 ms while another's might be 800
ms. With this knowledge in mind, we set \(r_{\alpha}\) to 1.0, encoding
the believe that the variation of individual baselines has a
characteristic scale of \(\sigma\).

We use a similar rational and \(g\)-prior structure on parameter
\(\nu\):

\begin{align*}
\nu|g_{\nu} &\sim \mbox{Normal}(0,\sigma^2g_{\nu})\\
g_{\nu} &\sim \mbox{Inverse-$\chi^2$}(r^2_{\nu}).\\
\end{align*}

To complete the specification, we let \(g_\theta = \eta^2/\sigma^2\),
the standardized variance of the effects. As before, the prior on
\(g_\theta\) is \[
g_\theta \sim \mbox{Inverse-$\chi^2$}(r^2_{\theta}).
\] To set values of \(r_\nu\), we note that average effects in tasks
like these tend to be on the order of tens of milleseconds, and we use
50 ms, or 1/6th of \(\sigma\), as the scale in the inverse-\(\chi^2\)
prior (\(r^2_\nu = (1/6)^2\)). To set \(r_\theta\) we consider
individual variability around the mean effect. It is difficult to know
what the true variation of individual effects is because the observed
variability is confounded by trial noise. Our working assumption is that
the scale should be also on the order of tens of milleseconds. Moreover
we expect the variability to be no larger than the size of the average
effect. Using this rational, we set the scale on the standard deviation
of individual differences to be 30ms, or 1/10th of \(\sigma\)
(\(r^2_\theta = (1/10)^2\)).

Figure 4B shows the marginal priors for two individuals' effects
\(\theta_i\). The correlation in the marginal priors arises from the
hierarchical nature of the specification in the unstructued and positive
models. The variability in \(\nu\) is shared and induces the
correlation, and the degree of correlation is a function of the amount
of shared variance, reflecting the setting of \(r_\nu\), and the amound
of unique variance, reflecting the setting of \(r_\theta\).

It is important to know how these substantive choices for setting
\(r_\alpha\), \(r_\nu\), and \(r_\theta\) affect model comparison
results. We visit this issue after presenting the results of the
real-world application.

\begin{figure}

{\centering \includegraphics{draft1_files/figure-latex/marginals-1} 

}

\caption{Marginal prior distributions for two different individuals. Values are shown in time units (ms) and in effect-size units (relative to $\sigma = 300$ ms). A. Distribution of $\alpha$ parameters (intercepts), B. Distribution of $\theta$ parameters (effects). Correlation in B comes from the hierarchical structure on effects.}\label{fig:marginals}
\end{figure}

\section{Model Comparison}\label{model-comparison}

A critical question is how to state evidence for the four
theoretically-informed models. One popular approach is to use Bayes rule
to compare models, and the resulting model-comparison statistic is the
\emph{Bayes factor} (Edwards, Lindman, \& Savage, 1963; Jeffreys, 1961;
Kass \& Raftery, 1995; Laplace, 1986; Rouder et al., 2009). Bayes rule
for the comparison of any two models, \(\calM_a\) vs. \(\calM_b\), is

\begin{equation}\label{bayesrule}
\frac{P(\calM_a \mid \bfY)}{P(\calM_b \mid \bfY)} = \frac{P(\bfY \mid \calM_a)}{P(\bfY \mid \calM_b)} \times \frac{P(\calM_a)}{P(\calM_b)},
\end{equation}

where the term on the left-hand side is the posterior odds for the
models, and the term on the far right,
\(\frac{P(\calM_a)}{P(\calM_b)}\), is the prior odds. The term
\(\frac{P(\bfY \mid \calM_a)}{P(\bfY \mid \calM_b)}\) is the Bayes
factor, and it describes how the data have changed the analysts'
beliefs. We, along with several others, argue the Bayes factor is the
\emph{evidence} from data for competing models, and is the appropriate
target of inquiry. See Edwards et al. (1963), Jeffreys (1961), and
Morey, Romeijn, \& Rouder (2016) for expanded discussions.

The numerator and denominator of the Bayes factor describe the
probability of observing data conditional on model \(\calM_a\) and model
\(\calM_b\), respectively. These probability statements may be termed
the \emph{predictions} of the respective models. They may be expressed
for all possible data points before the data are observed. The Bayes
factor therefore denotes how well the observed data are predicted under
one model relative to another. In practice, the Bayes factor has two
roles: It describes the evidence as change of beliefs and describes how
well the models predicted the observed data. The equivalence of these
roles leads to the deep insight that in Bayesian model comparison,
evidence for competing models is the ratio of how well each predicts the
observed data (Rouder, Morey, Verhagan, Swagman, \& Wagenmakers, 2016).

To display the predictions of the models, we focus on two hypothetical
individuals much as we did in specifying the models. The left column in
Figure 3 shows the model specification for set values of \(\nu\) and
\(\eta\). However, when marginalized across \(\nu\) and \(\eta\), there
is correlation among individual effects as shown in Figure 4B. The right
column of Figure 3 shows the corresponding predictions for sample
effects, which shows the correlation. Panel \(A_2\), shows the predicted
sample effects for the unstructured model \(\calM_u\) from panel
\(A_1\); panel \(B_2\) shows the predicitons for the positive-effects
model, \(\calM_+\); panel \(C_2\) shows the predicted individual effects
for the common-effect model, \(\calM_1\) and, finally, \(D_2\) shows the
distribution of predicted effects for the null model, \(\calM_0\). The
more constraint within a model, the greater the predictive density on
concordant points and the less density on discordant ones.

Once the predictions are derived, model comparison is as simple as
comparing these predictive densities at observed values. Suppose the
individuals have observed effects of \(d_1 = 150\) ms and \(d_2 = 100\)
ms. These values are shown by the point in panels \(A_2\) through
\(D_2\). The density for \((150, 100)\) under the unstructured model
\(\calM_u\) is 0.00103, the density under the positive-effects model,
\(\calM_+\), is 0.00204. The ratio of these two densities is 2-to-1 in
favor of \(\calM_+\). This ratio is a Bayes factor and is denoted
\(B_{+u}\). The comparison of all models is done in the same manner.

Although Bayes factors are conceptually simple, they are often
computationally inconvenient in real-world applications with mixed
models. The target quantity, the probability of data conditional on a
model, is rexpressed using The Law of Total Probability as

\begin{equation} \label{bfInt}
P(\bfY \mid \calM) = \int_{\bfxi \in \Xi} P(\bfY|\bfxi)P(\bfxi)d\bfxi,
\end{equation}

where \(\bfxi\) is a vector of parameters from parameter space \(\Xi\).
The terms in the integrand are straightforward to compute. The
difficulty comes from evaluating the integral itself. The integration is
multidimensional ranging over the dimensionality of the parameters.
There are often no closed form solutions, and brute-force numerical
methods fail in large-data settings because the to-be-integrated
multivariate function is very peaked and narrow. The numerical method
must find the \enquote{needle in the haystack.} Often, it is difficult
to assess whether the outputs approximate the true value of the
integral. As a result, obtaining computationally convenient Bayes factor
algorithms in mixed settings remains a timely and topical in Bayesian
research.

In the previous section, we present a \(g\)-prior specification. These
types of specifications were first proposed by Zellner \& Siow (1980)
and subsequently studied by several others including Bayarri \&
Garcia-Donato (2007), F. Liang et al. (2008), Rouder et al. (2012),
Zellner (1986). The particular form we adopt was first proposed by
Rouder et al. (2012) who developed models with multiple \(g\)-parameters
for mixed ANOVA designs with fixed and random effects. The key advantage
of this form, and the reason we use it here, is that we can ensure
accurate evaluation of the needed integrals.

Integration proceeds as follows: The full vector of parameters is the
concatination of the \(g\)-parameters, denoted
\(\bfg=(g_{\alpha}, g_\nu, g_\theta)\), and the remaining parameters,
denoted \(\bflambda=(\bfalpha, \bftheta, \mu, \nu, \sigma^2)\), where
\(\bfalpha\) and \(\bftheta\) are the collection of individual intercept
and slope parameters, respectively. With this notation \[
P(\bfY \mid \calM) = \int_{\bfg} \int_{\bflambda} P(\bfY|\bfg,\bflambda) P(\bflambda) d\bflambda P(\bfg) d\bfg 
\] The inner integral, denoted
\(T(\bfg) = \int_{\bflambda} P(\bfY | \bfg,\bflambda) P(\bflambda) d\bflambda\)
may be solved analytically. The solution is developed in Rouder et al.
(2012, p.~361-362) and implemented in Morey and Rouder's BayesFactor
package for R (Morey \& Rouder, 2015). Hence, \[
P(\bfY \mid \calM) = \int_{\bfg} T(\bfg) P(\bfg)d\bfg.
\] This integral is over only three dimensions in the current design.
More importantly, \(T(\bfg)\) varies smoothly across the the paramater
space of \(\bfg\), and the evaluation may be performed to high precision
by Monte Carlo sampling (see Rouder et al., 2012 for expanded analysis).
This approach in implemented in BayesFactor package using the nWayAOV
command separately for the unstructured model, the common-effect model,
and the the null model.

Unfortunately, for the positive model, it is not possible to integrate
\(\bflambda\) analytically due to the range restrictions on
\(\bftheta\). Instead, we follow the \emph{encompassing} approach
discussed by Klugkist and colleagues (Klugkist \& Hoijtink, 2007;
Klugkist, Kato, \& Hoijtink, 2005). In our case, where the prior on the
positive model is a truncated version of the prior on the unstructured
model, a simple counting approach within MCMC-sampling works well. The
main idea is as follows:

As seen in (\ref{bayesrule}), the Bayes factor is the ratio of the
posterior odds to the prior odds. These odds may be given by \[
\frac{P(\calM_+|\bfY)}{P(\calM_u|\bfY)} = P(\bftheta>\bf0|\bfY,\calM_u),
\] and \[
\frac{P(\calM_+)}{P(\calM_u)} = P(\bftheta>\bf0|\calM_u).
\] where \(\bftheta>\bf0\) refers to event that each element of
\(\bftheta\) is greater than zero. The Bayes factor, the ratio, is given
by \[
B_{+u} = \frac{P(\bftheta>0|\bfY,\calM_u)}{P(\bftheta>0|\calM_u)}.
\] Restated, the Bayes factor is the posterior probability that all
effects are positive relative to the prior probability of the same
event.

Computation of these probabilities is straightforward in MCMC sampling.
Let \(\bftheta[m]\) denote a vector of samples on the \(m\)th iteration
under the unstructured model. The \(m\)th iteration is considered
evidential of the positive-effects model if all \(I\) elements of
\(\bftheta[m]\) are positive and not otherwise. Let \(n_1^+\) be the
number of evidential iterations conditional on data, and let \(n_0^+\)
be the same from the prior. Then, the Bayes factor is, \[
B_{+u} = \frac{n_1^+}{n_0^+}.
\]

Additional derivations and development of this encompassing approach may
be found in Klugkist et al. (2005). To compute the Bayes factor of the
positive-effect model to the remaining models, we use the well-known
transitivity of Bayes factors (Rouder \& Morey, 2012).

\section{Seven Data Sets}\label{seven-data-sets}

\begin{figure}

{\centering \includegraphics{draft1_files/figure-latex/stroop_plot_model_comp-1} 

}

\caption{{\footnotesize Empirical and Bayesian analyses for the Stroop paradigm (Data Sets 1-3.) Each row shows one data set. The left column shows the ordered observed effects, $d_i$, as points. The shaded area denotes the associated 95\% confidence intervals, and the dashed line is the mean effect. The right column shows the ordered Bayesian estimates of individual effects, $\theta_i$. The points are the estimates from the unstructured model. The red line is the estimate from the constant-effect model. The observered effects, $d_i$ are included as grey line for comparison. In Panel $\mbox{A}_2$, the Bayesian analysis for Data Set 1, there is a sizable difference between the observed and estimated effects, and this difference is indicative of hierarchical shrinkage. Almost all the variability in the individual effects is seemingly from sample noise.}}\label{fig:stroop_plot_model_comp}
\end{figure}

\begin{figure}

{\centering \includegraphics{draft1_files/figure-latex/simon_plot_model_comp-1} 

}

\caption{{\footnotesize Empirical and Bayesian analyses for the Simon paradigm (Data Sets 4-6). The figure has the same format as Figure 5. The left column shows the ordered observed effects, $d_i$. The right column shows the ordered Bayesian estimates of individual effects, $\theta_i$.}}\label{fig:simon_plot_model_comp}
\end{figure}

\begin{figure}

{\centering \includegraphics{draft1_files/figure-latex/flanker_plot_model_comp-1} 

}

\caption{{\footnotesize Empirical and Bayesian analyses for the Eriksen flanker paradigm (Data Set 7). The figure has the same format as Figure 5. The left column shows the ordered observed effects, $d_i$. The right column shows the ordered Bayesian estimates of individual effects, $\theta_i$. The mean effect in panel $\mbox{A}_1$ is very close to zero. The parameter estimates for the unstructured model in panel $\mbox{A}_2$ are close to zero as well, indicating that almost all the variability in the individual effects is seemingly from sample noise.}}\label{fig:flanker_plot_model_comp}
\end{figure}

To demonstrate the need for the development of models like the above, we
use them to assess the structure of effects in seven existing data sets
that cover three common experimental tasks with massively repeated
designs. The tasks are common inference phenomena. The first phenomenon
is the aforementioned Stroop interference effect. We analyzed three data
sets, one from Von Bastian et al. (2015) and two from Pratte, Rouder,
Morey, \& Feng (2010).

The second phenomenon we analyzed is Simon interference (Simon, 1969).
In a Simon interference task, participants are asked to identify a
property of a stimulus, say its color. There are always two choices, and
say for the color example, patches might be presented in red or green.
Participants press buttons either on the left or right, and an example
might be to press a left key for red and a right key for green. The
interference comes from the spatial placement of the patch. Patches are
placed on the left or right: a congruent patch in this case is a red
patch on the left and the congruency is between the spatial placement
and the left-key response. Likewise a green patch on the right is
congruent. Incongruent patches are red patches on the right or green
patches on the left because the side of stimulus placement is opposite
(contralateral to) the side of the response. Simon effects are
relatively small, say about 30 ms, but the phenomenon is nonetheless
robust and well studied (C. H. Lu \& Proctor, 1995). We analyzed three
existing Simon interference data sets including one from Von Bastian et
al. (2015) and two from Pratte et al. (2010).

The third phenomenon we analyzed is an Eriksen flanker task (B. A.
Eriksen \& Eriksen, 1974) from Von Bastian et al. (2015). In the Eriksen
flanker task, the interference comes from distractors that are placed
around a target stimulus. For example, participants identify a centrally
located character, e.g.~a consonant like \enquote{H}. Those targets are
surrounded by distractors that could either match the target character,
say another consonant \enquote{K}, or they mismatch, say a vowel like
\enquote{E}.

Table 1 provides an overview over all seven data sets. For the
derivations of the used \(F\)-statistic and the confidence intervals in
Figures 5 to 7, see Appendix A. The cleaning strategy for the RT data
for all seven data sets is provided in Appendix B.

\subsection{Data Set 1}\label{data-set-1}

Set 1 is the Stroop task data from Von Bastian et al. (2015). The task
used is commonly called a \emph{number Stroop task} (West, Jakubek,
Wymbs, Perry, \& Moore, 2005 refer to the task as counting task), and it
goes as follows: On each trial, participants saw a string of digits. In
each string, the digits were always replicates, say \enquote{22} or
\enquote{444}, and the lengths varied from one digit to four digits. The
participants task was to report the length, for example, the correct
report for \enquote{444} is 3. In the congruent condition, the length
and the digits matched; e.g., \enquote{22} and \enquote{4444.} " In the
incongruent condition, the length and digits mismatched, e.g.,
\enquote{44} and \enquote{2222.} Additionally, participants saw neutral
trials that consisted of unrelated symbols, e.g. \enquote{\#\#\#}. In
total, 121 participants each responded to 48 congruent, 48 incongruent
and 48 neutral trials. For the following analysis, we only used data
from the congruent and the incongruent conditions.

The mean effect and the individual effects of all 121 participants are
shown in Figure 5\(\mbox{A}_1\). The dashed line is the mean effect
\(\bar{d} = 65\) ms, which corresponds to an effect size of
\(d = 1.37\). The points show ordered individual effects, and the grey
outer surface denotes the 95\% confidence intervals for each individual
(see the Appendix for derivations). The graph shows the same information
as Figure 2. A one-way random-effects ANOVA (see Appendix for
derivations) reveals individual variability, \(F(120, 11003)= 1.30\),
\(p \approx 0.016\).

\subsection{Data Set 2}\label{data-set-2}

Set 2 was conducted by Pratte et al. (2010) (Experiment 1, Stroop task).
The task they used is a classical Stroop task: Participants were asked
to identify the color of the presented color words, e.g.~the word
\enquote{RED} presented in blue. In the congruent condition,
presentation color and word meaning matched, e.g. \enquote{BLUE}
presented in blue. In the incongruent condition, they did not match,
e.g. \enquote{RED} presented in blue. In the neutral condition, a
neutral word was presented in a color, e.g. \enquote{XXXX} presented in
blue. 38 participants responded to 168 trial for each of those
conditions. For the following analysis, we only used data from the
congruent and the incongruent condition.

The mean effect and the individual effects are shown in Figure
5\(\mbox{B}_1\). The dashed line is the mean effect \(\bar{d} = 91\) ms,
with a corresponding effect size of \(d = 1.81\). The points in the
figure show ordered individual effects, and the grey outer surface
denotes the 95\% confidence intervals for each individual. A one-way
random-effects ANOVA shows individual variability,
\(F(37, 11038) = 2.60\), \(p \approx 0\).

\subsection{Data Set 3}\label{data-set-3}

Set 3 was originally conducted by Pratte et al. (2010). The data used
for this analysis stems from the five Stroop task blocks of Experiment 2
of the original study. In this task, the stimuli were the words
\emph{left} and \emph{right}, presented on the left or the right side of
the screen. Participants were asked to identify the position of the word
while ignoring the meaning of the word. A congruent trial occured when
position of the word and word meaning corresponded; an incongruent trial
emerged when position and word meaning did not correspond. In total, 38
participants responded to 180 congruent and 180 incongruent trials.

Mean and individual effects are shown in Figure 5\(\mbox{C}_1\). The
dashed line represents the mean effect \(\bar{d} = 12\) ms, with a
corresponding effect size of \(d = 0.60\). The points in the figure show
ordered individual effects, and the grey outer surface denotes the 95\%
confidence intervals for each individual. A one-way random-effects ANOVA
reveals no significant individual variability, \(F(37, 12489) = 1.25\),
\(p \approx 0.14\).

\subsection{Data Set 4}\label{data-set-4}

Set 4 is the Simon task data from Von Bastian et al. (2015). In this
task, either a red or a green circle was presented to the participants.
The circles were either presented on the right or on the left side of
the screen. Participants were asked to respond with the left-arrow key
to green circles and with the right-arrow key to red circles. In the
congruent condition, position of the circle and response position match.
In the incongruent condition, position of the circle and response
position do not match. In total, 121 participants responded to 150
congruent trials and 50 incongruent trials.

Figure 6\(\mbox{A}_1\) shows mean and individual effects. The dashed
line shows the large mean effect \(\bar{d} = 11.90\) ms, with a
corresponding effect size of \(d = 2.22\). The points in the figure show
ordered individual effects, and the grey outer surface denotes each
individual's 95\% confidence interval. A one-way random-effects ANOVA
reveals individual variability, \(F(120, 23211) = 2.65\),
\(p \approx 0\).

\subsection{Data Set 5}\label{data-set-5}

Data Set 5 is from experiment 2 conducted by Pratte et al. (2010) (Simon
task). In this task, participants saw either a green or a red target
stimulus on each trial. These targets were either presented on the left
or the right side of the screen. The participants were instructed to
press a key with their right hand for green and another key with their
left hand for red stimuli. In a congruent trial, target position and
response position match. In an incongruent trial, target position and
response position mismatch. For this set, 38 participants completed 252
incongruent and 252 congruent trials.

Mean and individual effects are shown in Figure 6\(\mbox{B}_1\). The
dashed line is the mean effect \(\bar{d} = 17.16\) ms, with a
corresponding effect size of \(d = 0.72\). The points in the figure show
ordered individual effects. A one-way random-effects ANOVA shows
individual variability, \(F(37, 17267) = 1.82\), \(p \approx 0.002\).

\subsection{Data Set 6}\label{data-set-6}

This set is from experiment 1 conducted by Pratte et al. (2010) (Simon
task). In this task, the target stimuli were the words \enquote{left}
and \enquote{right}. These words were either presented on the left or
the right side of the screen. The participants were instructed to press
a key with their right hand for the word \enquote{right} and another key
with their left hand for the word \enquote{left}. In a congruent trial,
target position and word meaning match. In an incongruent trial, target
position and word meaning mismatch. In total, 38 participants completed
180 incongruent and 180 congruent trials.

Figure 6\(\mbox{C}_1\) shows mean and individual effects. The dashed
line represents the mean effect \(\bar{d} = 30.29\) ms, with a
corresponding effect size of \(d = 1.02\). The points in the figure show
ordered individual effects. A one-way random-effects ANOVA reveals
individual variability, \(F(37, 12190) = 2.29\), \(p \approx 0\).

\subsection{Data Set 7}\label{data-set-7}

Set 7 is the Eriksen flanker task in Von Bastian et al. (2015). In the
task used, participants saw a string of seven characters. All of these
characters were identical except for the one in the middle. The task was
to decide whether this centrally located character was a vowel, e.g.
\enquote{S}, or a consonant, e.g. \enquote{A} or \enquote{E}. The
flanking six characters were as well either vowels, consonants or
neither, e.g. \enquote{\#}. In the congruent condition, the flanking
characters match the central character, e.g. \enquote{AAAEAAA}. In the
incongruent condition, the flanking characters and the central character
mismatch, e.g. \enquote{SSSESSS}. In the neutral condition, the central
character was flanked by \enquote{\#}, e.g. \enquote{\#\#\#E\#\#\#}. In
total, 121 participants responded to 48 trials for each of those
conditions. For the following analysis, we used data from the congruent
and the incongruent condition.

The mean effect and the individual effects are shown in Figure 7A. The
dashed line represents the mean effect \(\bar{d} = 2.21\) ms, with a
corresponding effect size of \(d = 0.07\). The points in the figure show
ordered individual effects, and the grey outer surface denotes the 95\%
confidence intervals for each individual. A one-way random-effects ANOVA
shows no significant individual variability, \(F(120, 10973) = 0.98\),
\(p \approx 0.538\).

\section{Analyses and Results}\label{analyses-and-results}

To illusrate the advantages of the Bayesian modeling approach, we
analyze the seven data sets with the four Bayesian models: the
unstructured model, \(\calM_u\); the positive-effects model,
\(\calM_+\); the common-effect model, \(\calM_1\); and the null model,
\(\calM_0\).

\subsection{Parameter Estimation}\label{parameter-estimation}

\begin{figure}

{\centering \includegraphics{draft1_files/figure-latex/convergence_plot-1} 

}

\caption{Examination of mixing in MCMC chains for Data Sets 1 and 2. Panels A (Data Set 1) and C (Data Set 2) show snippets of chains for $\eta^2$, the slowest converging parameter, for the unstructured model. Panels B and D show the autocorrelation function for the same two cases. Mixing is quite good for Data Set 2 and acceptable for Data Set 1.}\label{fig:convergence_plot}
\end{figure}

Posterior distributions for all parameters in the unstructured, the
common-effect and the null models were sampled with Markov chain Monte
Carlo methods (MCMC). In all cases, conditional posterior distributions
of parameters may be derived straightforwardly from Bayes rule (Gelman,
Carlin, Stern, \& Rubin, 2004; Jackman, 2009; Rouder \& Lu, 2005).
Priors were chosen to leverage conjugacy, and consequently in almost all
cases posterior distributions may be sampled from known distributions as
Gibbs steps (Gelfand \& Smith, 1990). This approach is implemented in
the BayesFactor package in R (Morey \& Rouder, 2015).

\subsubsection{Mixing}\label{mixing}

MCMC chains are guaranteed to provide samples from the joint posterior
of all parameters in the large-sample limit. In practice, however,
outputs from successive iterations are often correlated. If this
correlation is severe, then there is no guarantee that the posterior has
been well explored. This problem of excessive correlation, when it
exists is known as a problem of \emph{mixing}. Chains without excessive
correlation from iteration to iteration are said to mix well, chains
with excessive correlation are said to mix poorly.

MCMC chains mixed well for the three estimated models. This rapid
convergence is expected here as the models are linear and not overly
saturated. The slowest converging parameters were the variances of the
effects, \(\eta^2\). Figure 8A and B provide a snippet of a chain and
the autocorrelation function, respectively, for this parameter for the
unstructured model in the worst case across all data sets (which was
from Data Set 1). As can be seen, mixing is acceptable even in this
worst case. The correlations here are not consequential for long runs of
several thousand iterations (our posterior estimates come from 20,000
iterations).\\The mixing in the same case for Data Set 2 was
considerably better. Figure 8C and D show mixing for this Data Set, also
is for \(\eta^2\). Even though this is the worst-case for the Data Set,
the mixing here is quite good.

\subsubsection{Results}\label{results}

The critical parameters are the individuals' effects, \(\theta_i\). The
posterior means of these parameters are shown in the right-hand columns
in Figures 5, 6, and 7 for the seven data sets. The dark grey lines that
have the largest spread are the observed effects, \(d_i\), and these are
included for comparison to the model estimates. The points are the
estimates \(\theta_i\) from the unstructured model. The red horizontal
line is the common effect estimate from the common-effect model
\(\calM_1\). Note that the right-hand columns in the figures do not have
the same scaling on the y-axis when comparing effects between different
paradigms.

There are three main findings:

\begin{enumerate}
\def\labelenumi{\arabic{enumi}.}
\item
  There is a sizable degree of shrinkage in all seven data sets. In all
  cases, much of the variation in the observed effects is due to sample
  variation at the trial level rather than true variation across
  individuals. This shrinkage is especially striking for Data Set 1, 3,
  5, and 7.
\item
  According to the unstructured model, most individuals have
  positively-values effects. This positivity holds even when the
  observed effects are negative. This positivity is a direct result of
  shrinkage to the overall mean, which is positive for Data Sets 1
  through 6.
\item
  The hierarchical models provide a reasoned estimate of individual
  variation while simultaneously accounting for sample noise at the
  trial level. The results are that there is dramatically less variation
  at the individual level. Consequently, the effect size, the magnitidue
  of the mean effect relative to the variation in individuals, is larger
  after accounting for trial-by-trial variation. For Data Set 1, for
  example, the observed effect size is 1.40 reflecting a mean effect of
  65 ms and an observed standard deviation around this mean of 47.37 ms.
  For the unstructured model estimates, in contrast, the effect size is
  9.70, and this increase reflects the decreased variation in
  individuals' effects. It is our belief that in within-subject designs,
  appropriately measured hierarchical model-based effect sizes are much
  larger than is typically reported. The observed and model-based effect
  sizes for all data sets are shown in Table 1.
\end{enumerate}

\subsection{Model Comparison}\label{model-comparison-1}

To compare the four models, we use the Bayes-factor approach discussed
previously. The values for all seven data sets are provided in Table 2.
The asterisk in each column marks the preferred model for each data set.
The values in the table are the Bayes factor between the respective
model and the preferred model. For example, for Data Set 1, the
positive-effects model is the preferred model. The table shows how much
worse the other models perform compared to the preferred one. The Bayes
factor for the runner-up, the common-effect model, over the
positive-effects model for Data Set 1 is 1-to-11. The Bayes factor for
the unstructured model over the positive-effects model is 1-to-12 for
the same data set. These Bayes factors indicate that the two models,
\(\calM_1\) and \(\calM_u\), predict the data worse than the
positive-effects model by about an order. The Bayes factor for the null
model is much worse, \(B_{0+} = 10^{-62}\).

There are two major findings:

\begin{enumerate}
\def\labelenumi{\arabic{enumi}.}
\item
  The preferred model varies across the tasks. For the Stroop tasks, the
  positive-effects and the common effect models are preferred indicating
  that all people Stroop in the same, usual direction. For the Simon
  tasks, in contrast, the unstructured model was slightly preferred for
  two of the three sets. This result provides modest evidence that
  perhaps some people truly have reverse Simon effects, where spatially
  incompatible responses are speeded relative to spatially compatible
  ones. For the Flanker task, there was very little effect in the data
  and, not surprisingly, the null model was preferred.
\item
  In Data Sets 3 and 7 there is no evidence for individual variablility.
  In these sets the mean effect is quite small and presumably any true
  individual differences, should they exist, would even be smaller.
  Hence, finding them would require larger numbers of trials per
  individual. The combination of the estimation results and the Bayes
  factor model comparisons. provide the following interpretation:
  Although there is far less variability in true individual effects than
  in observed effects, the degree of true variation is nonetheless
  substantial for five of the seven data sets. Also, the concordance
  between shrinkage in estimation and the Bayes factors is
  noteworthy---the greater the shrinkage, the better the relative
  performance of the common-effect model to the unstructured and
  positive-effects models.
\end{enumerate}

\section{Sensitivity to Prior
Settings}\label{sensitivity-to-prior-settings}

In analysis, it is necessary to specify the prior scales \(r_\alpha\),
\(r_\nu\), and \(r_\theta\), which are the standardized scales on
individual intercepts, the population mean effect, and individual
variation around this mean. Of these three, the setting of \(r_\alpha\)
is relatively inconsequential as individual intercepts are specified in
all four models. The other two settings are consequential as they define
the dimensionality of the models on \(\theta_i\). If, for example,
\(r_\theta\) is large, then there is little \emph{a priori} correlation
among the individual effects, and the dimensionality of this part of the
model is large. Likewise, if \(r_\theta\) is small, then all people have
nearly the same effect, and the dimensionality of this part of the model
is small.

Before exploring the effects of different settings of \(r_\nu\) and
\(r_\theta\), we provide some guidance as context. It may seem natural
to view dependence of the Bayes factor on these settings as difficult or
undesirable. We think, however, the situation is far more nuanced. In
hierarchical models, the settings are more akin to model specification
inasmuch as they set the dimensionality of the model on individual
differences. Setting the scale too small makes the unstructured model
resemble too closely the common-effect model; setting the scale too
large makes the unstructured model needlessly complex. To us,
dimensionality should be a key consideration in specification. And where
we would expect inference about a model to depend quite heavily on its
specification including its complexity, we expect the Bayes factors to
depend markedly on these settings. We first highlight the dependency and
then provide some interpretation.

Table 3 shows the Bayes factors for different settings of \(r_\nu\) and
\(r_\theta\) for Data Set 1. The first line provides the settings we
used for our analysis, \(r_\nu = 1/6\) and \(r_\theta = 1/10\). If we
assume 300 ms of overall variability, the settings translate into about
50 ms and 30 ms, respectively. For these settings, the estimates of
\(\theta_i\) (posterior means) from the unstructured model have a
standard deviation of 7 ms, which is a data-driven estimate of the true
individual variability under the assumption that individuals truly vary.
One way to view this state is that we had expected around 30 ms of
individual variation \emph{a priori} but observed only 7 ms in the data.
Bayes factors here favor the positive-effects model over the
unstructured and common-effect model indicating that while these 7 ms
are small, they are detectable as individual variation.

The next two lines show the case that the settings are either halved or
doubled. In the second line, we expect a smaller effect overall and
smaller individual variation around this effect. We see here
attenuation---the estimated individual variation is 4 ms---and still
this small estimate is detectable. The next line shows the case of
doubling the expectations; here we see that although all the conclusions
remain, the evidence for the positive-effects model is not as great as
before. It predicts the data 2.7 times better than the common-effect
model. If we continue to increase \(r_\theta\), the scale of individual
variation, the common-effect model gains, and is preferred. Indeed, as
shown in the fifth and the seventh lines, this trend can be quite
extreme.

To us, these dependencies are reasonable and desirable. First, in the
range that we consider reasonable, the Bayes factors have modest
variation. For example, if we take lines two and three as defining a
reasonable range for effect scales, the Bayes factors between the
winning model, the positive-effects model, and the common-effect model
varies from 2.7-to-1 to 14.2-to-1. While this variation is moderate, it
covers in our opinion the full range of variation across reasonable
researchers. This amount of variation may be small when compared to
variation from other subjective elements in research. Second, if
researchers make unreasonable commitments, then the Bayes factors change
dramatically. An \(r_\nu\) or \(r_\theta\) of 1.0 is unreasonable
because nobody expects to find an effect that is equal in size to the
variation in response times across repeated trials. If this were so, no
sane researcher would use 50 or 100 repeated trials per person per
condition. In this case, the unreasonable specification leads to
predicted effects that are far too variable compared to the data.
Consequently, the Bayes factors indicate that these unreasonable
specifications are too complex.

\section{General Discussion}\label{general-discussion}

\begin{figure}

{\centering \includegraphics{draft1_files/figure-latex/deltaplot-1} 

}

\caption{Delta plots for Data Sets 1-6. A. The three Stroop task data (Sets 1-3). All Stroop interference data sets show a positive slope as is often found in these kinds of tasks. B. The Stroop task data (Sets 4-6). Two of the three data sets show a negative slope.}\label{fig:deltaplot}
\end{figure}

In this paper we develop a set of Bayesian mixed models for assessing
equality and order constraints in simple experimental paradigms. These
models range through four steps: the simplest null model, the model
where all individual share a common effect, the model where individual
effects vary but share a common direction, and an unstructured model
that places no such constraints on individual variability. The models
developed here lead to computationally convenient Bayes factors, whoch
serve as principled measures of the strength of evidence for the varying
degrees of constraint in the data.

From a psychological perspective, perhaps the most important
consideration for well-established effects is whether the order
constraint holds. In the Stroop case, for example, we ask whether all
individuals have true Stroop effects in the same direction where
congruent colors are named more quickly than incongruent ones. This
constraint is compatible with the leading explanation that Stroop
interference results from the fact that reading is quick and automatic
for competent readers (MacLeod, 1991). And assuming that all of the
college participants are competent readers, then the constraint should
hold. Indeed, we suspect that many tasks will plausibly obey an order
constraint. The interpretation is that they are mediated by nearly
universal, automatic processes that do not admit a reverse ordering. We
do not expect, however, that all tasks will order as such, and in those
that do not, it is reasonable to search for differing strategies and
processing among participants.

\subsection{Stroop, Simon, and Eriksen
Interference}\label{stroop-simon-and-eriksen-interference}

In the course of development, we chose Stroop, Simon, and Flanker
interference as our first application. We are familiar with these types
of interference (Pratte et al., 2010; Rouder, Yue, Speckman, Pratte, \&
Province, 2010; Speckman, Rouder, Morey, \& Pratte, 2008) and were
fairly certain \emph{a priori} that Stroop intereference would obey the
order constraint. Hence, we are not at all surprised that in all three
Stroop data sets, the positive-effects, or positive common-effect models
were preferred.

Stroop effects follow a common pattern when response time distributions
are considered (Rouder et al., 2010). Figure 9 shows \emph{delta plots}
for Data Sets 1-6. A delta plot, first introduced by {De Jong}, Liang,
\& Lauber (1994), is a rotated version of a QQ plot (Zhang \& Kornblum,
1997). Each point denotes a RT percentile rank, and for the nine points
displayed on each line, the ranks are the
\(10^{th},20^{th},\ldots,90^{th}\) percentiles. The x-axis value of a
point is the average RT at that percentile across the congruent and
incongruent condition; the y-axis value is the difference, that is, the
effect.\footnote{Following Zhang \& Kornblum (1997), we compute the
  points on the delta plot lines as follows: Let \(\bar{y}^p_j\) denote
  the average reaction time for the \(p\)th percentile and the \(j\)th
  condition across individuals and trials. This is a reasonable values
  as long as the assumption holds that the individuals' RT distributions
  do not vary in shape. We can now compute the average RT for the
  \(p\)th percentile, \(\frac{\bar{y}^p_1 + \bar{y}^p_2}{2}\) and the
  average \(p\)th percentile effect, \(\bar{y}^p_2 - \bar{y}^p_1\). For
  a delta plot, we plot those values against each other. If the slope of
  the line is positive, the fastest responses have the smallest, or even
  a negative effect. If the slope is negative, the fastest responses
  have the biggest effect.} There is a common pattern to these plots
where effects tend to be smaller for lower percentiles (the faster
responses) and larger for higher percentiles (the slower responses)
(Luce, 1986; Rouder et al., 2010; E. J. Wagenmakers \& Brown, 2007).
Indeed, Stroop interference effects in the literature show this common
pattern (see Pratte et al., 2010), and it is present in the analyzed
data sets as well (Figure 9A).

Simon interference has intrigued researchers for decades because the
phenomenology seems idiosyncratic. Delta plots commonly have a markedly
different pattern. They start positive but have a negative slope. In
several experimental reports, they actually cross the centerline
implying that the slowest responses to incongruent stimuli are faster
than the slowest responses to congruent ones! The negative slope has
been observed regularly (e.g.~styrkowiec2013space; Burle, {van den
Wildenberg}, \& Ridderinkhof, 2005; {De Jong} et al., 1994). In Figure
9B, delta plots of the three Simon interference data sets are shown. Two
of them show the typical negative-slope pattern. The reversal
characterizes Data Set 5 as well as data from Burle et al. (2005). There
is no other effect that we are aware of that has this negative-slope
pattern. Based on this result as well as electrophysiology results, the
leading theory of Simon intereference is that it reflects a quick
automatic process followed by a separate, slower, compensation process
(Ridderinkhof, 1997).

The plausible presence of two opposing processes for Simon interference
sets up the possiblity that individual mean Simon effects are not order
constrained. Most participants may have a larger early automatic
positive component, but some may have a larger, later negative
component. Hence, the unstructured model may provide the best account.

We considered predictions before analysis and speculated that Stroop
interference would obey the order constraint---indeed, everyone Stroops.
We decided not to speculate about Simon interference beforehand. If the
order constraint held, it could be that the positive component was
always larger than the negative one. Likewise, if it did not, as it may
not here, then there is individual variation in the relative sizes of
the components.

The combination of results do lead to a new conjecture. There may be a
profitable link between the delta plot pattern and the order
restriction. Perhaps whenever the slope is negative---an indication for
two opposing processes---the order constraint may be violated.

The null Eriksen interference result is surprising because Eriksen
effects are well known and often reported. We did find a Eriksen effect
in accuracy but, as shown, not in RT. Given the prevalence of RT effects
in the literature, and that we did not collect these data, we do not
further speculate on the null RT result.

\subsection{The Interpretability of a Common
Effect}\label{the-interpretability-of-a-common-effect}

Two of our four models, the null model and the common effect model, are
novel in that they specify no individual differences whatsoever. To our
knowledge, individual differences researchers rarely consider a lack of
individual differences. In this regard, these models serve as important
controls. Indeed, in our data, these models with no individual
differences outperformed the others in two of the seven data sets.

The null model seems plausible or at least theoretically useful as a
bound on human behavior. There are simply some effects that do not occur
for anyone, perhaps the ability to predict lottery outcomes above
chance. Hence, this model is fairly interpretable. But what about the
common-effect model? The notion that there is a natural constant for
Stroop, that each person has the same exact effect, say 60 ms, is not
too believable. If we deem this model unbelievable, then how may we
interpret the event that it outperforms the other models?

Perhaps the most fruitful position is to interpret the strong
performance of this model in the context of the experimental design.
When the data are few in number, then simpler models are preferred to
more complicated ones. If researchers are interested in studying the
structure of individual variation, then they need to employ designs with
larger sample sizes. The analyses here indicate that, from a design
perspective, the number of trials per individual per condition is
critical. When this number is small, it is difficult to separate true
individual variation from sample noise, and in light of this
difficulity, the common-effect model provides a more parsimonious
description (as it should). As this number becomes larger, the
separation of sources of variability is more stable, and evidence for
true individual variation may be observed through model comparison.
Hence, we recommend the individual-difference researchers to consider
the common-effect model as a check that designs have sufficient trials
per individual to resolve true individual differences.

\subsection{Computational
Considerations}\label{computational-considerations}

The main computational issue in analyzing these models is computing
Bayes factors. The Bayes factor is the relative probability of data, and
computing this probability requires accurate integration across all
parameters with respect to the priors. To make the integration
convenient, we used a \(g\)-prior setup that is common in linear models
(Bayarri \& Garcia-Donato, 2007; F. Liang et al., 2008). In this setup,
models are placed on standardized effect sizes rather than on effects
themsleves. With this setup, all parameters save the variabilities on
effect sizes (\(g_\alpha, g_\nu, g_\theta\)) may be integrated in closed
form, greatly simplifying the problem.

The \(g\)-prior setup was not our first choice. Instead, we placed
models on effects themselves, and tried a brute-force approach known as
the \emph{Savage-Dickey Density Ratio} (Dickey, Lientz, \& others, 1970)
for computing the Bayes factor between the common-effect and
unstructured models. This method is precendented and recommended in
psychology (Morey, Rouder, Pratte, \& Speckman, 2011; E.-J. Wagenmakers,
Lodewyckx, Kuriyal, \& Grasman, 2010; Ruud Wetzels, Grasman, \&
Wagenmakers, 2010). Unfortunately, the posterior density ratio estimates
are too unstable to compute accurately with MCMC outputs. Samples varied
by some 40 orders of magnitude and convergence of the running mean was
not achieved even with ten million MCMC iterations.

\subsection{Limitations}\label{limitations}

Although the \(g\)-prior approach with symbolic integration of most
parameters is suitable here, there are two substantial limitations
affecting future generalizations. One of our goals at the outset of this
project was to include mixture models, say those where each individual
had identically no effect or came from a positive distribution. These
types of mixture models are common in Bayesian analysis and go under the
moniker of \enquote{spike-and-slab} priors (George \& McCulloch, 1993).
The usual goal with these models is to categorize each individual as
being in the spike, that is having no effect, or being in the slab, that
is, having an effect. And a separate Bayes factor is computed for each
individual. To our knowledge, there has been no consideration of the
more natural goal in this setting---to assess whether the constraints
from the mixture model, taken globally, provide a good description of
the structure in the data. We seek to compare the mixture model as a
whole to the positive-effects model, the equivalent model without the
spike. Whether the approach here generalizes to the mixture case remains
unclear.

Another of our goals was developing realistic three-parameter lognormal
models on response time such as those in (Rouder, Province, Morey,
Gomez, \& Heathcote, 2015). Currently, we use the normal with constant
variance. The lognormal is desirable because it captures the fact that
response time distributions are skewed and that effects tend to be
manifest in the scale rather than in the shift or shape of the
distribution. It is not clear, however, how to integrate log-normal
parameters in the three-parameter version.

Although these generalizations do not yield convenient closed form
solutions for the integration, there are still avenues for future
development. Fortunately, the computational toolbox for Bayes factors in
mixed models is sizable and growing. Candidates for future work include
bridge sampling (Meng \& Wong, 1996), importance sampling (Kass \&
Raftery, 1995) and Laplace approximation (Kass \& Raftery, 1995;
Raftery, 1996). Hopefully, progress in computational issues will allow
for useful generalizations of the models developed here.

\clearpage

\section{Appendix}\label{appendix}

\subsection{Appendix A.}\label{appendix-a.}

In this appendix we derive confidence intervals and the \(F\)-test. Let
\(Y_{ijk}\) denote the \(k\)th response for the \(i\)th participant in
the \(j\)th conditions, \(i=1,\ldots,I\), \(j=1,2\),
\(k=1,\ldots, K_{ij}\).

\subsubsection{Individuals' Confidence
Intervals}\label{individuals-confidence-intervals}

The goal here is to derive confidence intervals for each individual in
isolation without any model-based pooling. Let \(M_{i1}\) and \(M_{i2}\)
denote the sample means in the congruent and incongruent conditions,
respectively, for the \(i\)th participant. Let \(V_{i1}\) and \(V_{i2}\)
denote the corresponding sample variances. The sample effect, \(d_i\),
is \(d_i=M_{i2}-M_{i1}\) and standard error of this effect, denoted
\(s_{di}\) is \[
s_{di} = \left(\frac{V_{i1}}{K_{i1}}+\frac{V_{i2}}{K_{i2}}\right)^{1/2}.
\] This standard error may be used to compute individuals' CIs in the
usual manner (Hays, 1994).

\subsubsection{\emph{F}-test}\label{f-test}

The goal is to derive a one-way, random-effects \(F\)-test to assess
whether there is any variation across individuals. This derived value
can be interpreted as a proper \(F\) if equal sample sizes are assumed
(i.e.~equal trial number in congruent and incongruent conditions). The
effect, \(d_i=M_{i2}-M_{i1}\), is distributed as \[
d_i \sim \mbox{Normal}\left(\mu_{i2}-\mu_{i1},\frac{\sigma^2}{K_{i1}}+\frac{\sigma^2}{K_{i2}}\right),
\]\\where \(\mu_{i2}\) and \(\mu_{i1}\) are true conditions means for
the \(i\)th person, and \(\sigma^2\) is the variability in any
observation around its true mean. This expression may be expressed as

\begin{equation}\label{diffModel}
d_i \sim \mbox{N}\left(\mu_{i2}-\mu_{i1},\frac{\sigma^2}{K^*_i}\right),
\end{equation}

where \(K^*_i\) is the \emph{effective sample size} for the \(i\)th
individual: \[
K^*_i=\frac{K_{i1}K_{i2}}{K_{i1}+K_{i2}}.
\] Consequently, \[
\sqrt{K^*_i} d_i \sim \mbox{Normal}(\sqrt{K^*_i}(\mu_{i2}-\mu_{i1}),\sigma^2).
\] Let \(d^*_i=d_i\sqrt{K^*_i}\). Consider the case that
\(\sqrt{K^*_i}\) is approximately constant for all individuals, that is
the design is approximately balanced. Then, under the null that
\(\mu_{i2}-\mu_{i1}\) is constant for all individuals, the following
expression is a between-participant estimator of \(\sigma^2\): \[
s^2_1 =\frac{\sum_i(d^*_i-\bar{d^*})^2}{(I-1)},
\] where \(\bar{d^*}\) is the mean of all \(d^*_i\). The
degrees-of-freedom associated with this estimator is \(I-1\).

The expression for within-subject variation follows from the usual
considerations: \[
s^2_2 = \frac{\sum_{ijk} (Y_{ijk}-M_{ij})^2}{\sum_{ij}(K_{ij}-1)}=\frac{\sum_{ijk} (Y_{ijk}-M_{ij})^2}{N-IJ},
\] where \(N=\sum_{ij}K_{ij}\) is the total number of observations. The
degrees-of-freedom associated with this estimator is \(N-IJ\).

With these expressions, the \(F\) statistic is \[
F = \frac{s^2_1}{s^2_2},
\] which under the null is distributed as an \(F\) distribution with
\(\nu_1=I-1\) and \(\nu_2=N-IJ\) degrees-of-freedom. Of course, this
computation only holds for nearly balanced designs. Fortunately,
deviations from balance only occurs when participants make errors, which
is not that common here (see Appendix B).

\subsection{Appendix B.}\label{appendix-b.}

Neither of the authors discussed their strategies for cleaning the data
in the original articles, even though there were clear outliers in the
data. The cleaning code of Pratte et al. (2010) was available to us. The
authors used three relatively strict criteria to decide which data
points should be excluded and we followed all three criteria for Data
Sets 2, 3, 5, and 6 and two of the criteria for Data Sets 1, 4, and 7
(data from Von Bastian et al. (2015)). The criteria were

\begin{enumerate}
\def\labelenumi{\Roman{enumi}.}
\itemsep1pt\parskip0pt\parsep0pt
\item
  All incorrect trials were removed.\\
\item
  We removed all trials with RTs less than .2 sec on the grounds that
  these times are too fast to be related to the processes of interest.
  We removed all trials with RTs greater than 2 sec on the grounds that
  these times are too slow to be related to the processes of interest.\\
\item
  We removed the first five trials in each experimental block,
  accounting for the familiarization with the task.
\end{enumerate}

These removals comprised between 3 \% and 13 \% of the originally
collected trials in each Data Set. Table 4 shows those removals for all
Data Sets broken down for the three criteria. The data of all seven sets
and the cleaning code are available at
\href{https://github.com/PerceptionCognitionLab/data0/tree/master/contexteffects}{\url{https://github.com/PerceptionCognitionLab/data0/tree/master/contexteffects}}.

\clearpage

\section{References}\label{references}

\setlength{\parindent}{-0.5in} \setlength{\leftskip}{0.5in}

\begin{table}[tbp]
\begin{center}
\begin{threeparttable}
\caption{Characteristics of the Data Sets}
\begin{tabular}{lrrrrrrr}
\toprule
 & \multicolumn{7}{c}{\bf{Data Set}} \\
\cmidrule(r){2-8}
 & \bf{1} & \bf{2} & \bf{3} & \bf{4} & \bf{5} & \bf{6} & \bf{7}\\
\bf{Participants} & 121 & 38 & 38 & 121 & 38 & 38 & 121\\
\bf{Trials Per Cell} & 48 & 168 & 180 & $\approx$ 100 & 252 & 180 & 48\\
\bf{Mean Effect (ms)} & 65 & 91 & 12 & 79 & 17 & 30 & 2\\
\bf{Observed effect size} & 1.37 & 1.81 & 0.6 & 2.22 & 0.72 & 1.02 & 0.07\\
\bf{Model-based effect size} & 9.68 & 2.88 & 1.84 & 3.5 & 1.65 & 1.79 & 0.57\\
\bf{F-value} & 1.3 & 2.6 & 1.25 & 2.65 & 1.82 & 2.29 & 0.98\\
\bf{p-value} & 0.016 & 0 & 0.14 & 0 & 0.002 & 0 & 0.538\\
\bf{Preferred model (BF)} & $\calM_+$ & $\calM_+$ & $\calM_1$ & $\calM_+$ & $\calM_u$ & $\calM_u$ & $\calM_0$\\
\bottomrule
\end{tabular}
\begin{tablenotes}[para]
\textit{Note.} The model-based effect size is from the unstructured model (see text).  The $F$- and $p$-value are appropriate frequentist tests for individual differences (see Appendix A).  Data set 4 has unequal trial numbers per condition: 50 incongruent and 150 congruent trials.
\end{tablenotes}
\end{threeparttable}
\end{center}
\end{table}

\begin{table}[tbp]
\begin{center}
\begin{threeparttable}
\caption{Bayes factor model comparison}
\begin{tabular}{lccccccc}
\toprule
 & \multicolumn{7}{c}{\bf{Data Set}} \\
\cmidrule(r){2-8}
 & \bf{1} & \bf{2} & \bf{3} & \bf{4} & \bf{5} & \bf{6} & \bf{7}\\
$\calM_0$ & 1 to $10^{62}$ & 1 to $10^{75}$ & 1 to 379 & $\approx 0$ & 1 to $10^7$ & 1 to $10^{21}$ & $*$\\
$\calM_1$ & 1 to 11 & 1 to $10^{7}$ & $*$ & 1 to $10^{19}$ & 1 to 14 & 1 to 2784 & 1 to 8\\
$\calM_+$ & $*$ & $*$ & 1 to 2.97 & $*$ & 1 to 2 & 1 to 1.3 & $\approx 0$\\
$\calM_u$ & 1 to 12 & 1 to 9 & 1 to 1.57 & 1 to 11 & $*$ & $*$ & 1 to 27\\
\bottomrule
\end{tabular}
\begin{tablenotes}[para]
\textit{Note.} Asterisks mark the preferred model for each data set.  The remaining values are the Bayes factors between a model and the preferred model for each data set.
\end{tablenotes}
\end{threeparttable}
\end{center}
\end{table}

\begin{table}[tbp]
\begin{center}
\begin{threeparttable}
\caption{Sensitivity of Bayes factors to scale settings}
\begin{tabular}{lccrrr}
\toprule
  & \multicolumn{1}{c}{$r_\nu$} & \multicolumn{1}{c}{$r_\theta$} & \multicolumn{1}{c}{SD($\theta_i$)} & \multicolumn{1}{c}{$B_{1u}$} & \multicolumn{1}{c}{$B_{+u}$}\\
\midrule
1 & 0.167 (50ms) & 0.1 (30ms) & 7 ms & 1.04 to 1 & 11.63 to 1\\
2 & 0.08 (25ms) & 0.05 (15ms) & 4 ms & 0.83 to 1 & 11.82 to 1\\
3 & 0.33 (100ms) & 0.2 (60ms) & 10 ms & 3.41 to 1 & 9.34 to 1\\
4 & 0.33 (100ms) & 0.33 (100ms) & 13 ms & 19.22 to 1 & 10.04 to 1\\
5 & 1 (300ms) & 1 (300ms) & 22 ms & $10^7$ to 1 & 0.39 to 1\\
6 & 1 (300ms) & 0.1 (30ms) & 7 ms & 0.88 to 1 & 3.35 to 1\\
7 & 0.167 (50ms) & 1 (300ms) & 22 ms & $10^7$ to 1 & 1.67 to 1\\
\bottomrule
\end{tabular}
\begin{tablenotes}[para]
\textit{Note.} Sensitivity analysis of Bayes factor computation for Data Set 1. Shown are the Bayes factors and the standard deviations for estimates of $\theta_i$ from the unstructured model. For ease of comparison, the values in parentheses show translations into variability when an overall standard deviation of 300ms is assumed. The first row shows the settings used for the analysis.
\end{tablenotes}
\end{threeparttable}
\end{center}
\end{table}

\begin{table}[tbp]
\begin{center}
\begin{threeparttable}
\caption{Percentage of excluded observations.}
\begin{tabular}{lrrrrrrr}
\toprule
 & \multicolumn{7}{c}{\bf{Data Sets}} \\
\cmidrule(r){2-8}
 & \bf{1} & \bf{2} & \bf{3} & \bf{4} & \bf{5} & \bf{6} & \bf{7}\\
\bf{Criterion I.} & 0.42 & 2.43 & 1.02 & 0.07 & 0.48 & 0.75 & 0.4\\
\bf{Criterion II.} & 2.78 & 3.98 & 0.85 & 3.04 & 2.38 & 3.16 & 3.14\\
\bf{Criterion III.} & --- & 8.8 & 8.33 & --- & 8.33 & 8.33 & ---\\
\bf{Total} & 3.19 & 12.95 & 8.15 & 3.09 & 9.45 & 10.34 & 3.45\\
\bottomrule
\end{tabular}
\begin{tablenotes}[para]
\textit{Note.} The three criteria for exclusion are: I. All incorrect trials. II. All trials with RTs less than .2 s and greater than 2 s. III. The first five trials in each experimental block.
\end{tablenotes}
\end{threeparttable}
\end{center}
\end{table}

Amodio, D. M., Harmon-Jones, E., Devine, P. G., Curtin, J. J., Hartley,
S. L., \& Covert, A. E. (2004). Neural signals for the detection of
unintentional race bias. \emph{Psychological Science}, \emph{15},
88--93.

Aust, F., \& Barth, M. (2015). \emph{Papaja: Create aPA manuscripts with
rMarkdown}. Retrieved from \url{https://github.com/crsh/papaja}

Balota, D. A., Cortese, M. J., Sergent-Marshall, S. D., Spieler, D. H.,
\& Yap, M. J. (2004). Visual word recognition of single-syllable words.
\emph{Journal of Experimental Psychology: General}, \emph{133},
283--316.

Bates, D., \& Maechler, M. (2016). \emph{Matrix: Sparse and dense matrix
classes and methods}. Retrieved from
\url{https://CRAN.R-project.org/package=Matrix}

Bayarri, M. J., \& Garcia-Donato, G. (2007). Extending conventional
priors for testing general hypotheses in linear models.
\emph{Biometrika}, \emph{94}, 135--152.

Burle, B., {van den Wildenberg}, W. P. M., \& Ridderinkhof, K. R.
(2005). Dynamics of facilitation and interference in cue-priming and
Simon tasks. \emph{European Journal of Cognitive Psychology}, \emph{17},
619--641.

Chauss{é}, P. (2010). Computing generalized method of moments and
generalized empirical likelihood with R. \emph{Journal of Statistical
Software}, \emph{34}(11), 1--35. Retrieved from
\url{http://www.jstatsoft.org/v34/i11/}

Cohen, J. (1994). The earth is round (\(p<.05\)). \emph{American
Psychologist}, \emph{49}, 997--1003.

Coren, S. (1993). \emph{The left-hander syndrome: The causes and
consequences of left-handedness}. New York: The Free Press.

{De Jong}, R., Liang, C. C., \& Lauber, E. (1994). Conditional and
unconditional automaticity: A dual-process model of effects of spatial
stimulus-response concordance. \emph{Journal of Experimental Psychology:
Human Perception and Performance}, \emph{20}, 731--750.

DeGroot, M. H. (1982). Lindley's paradox: Comment. \emph{Journal of the
American Statistical Association}, \emph{77}(378), 336--339. Retrieved
from \url{http://www.jstor.org/stable/2287246}

Dickey, J. M., Lientz, B., \& others. (1970). The weighted likelihood
ratio, sharp hypotheses about chances, the order of a markov chain.
\emph{The Annals of Mathematical Statistics}, \emph{41}(1), 214--226.

Dinapoli, N., \& Gatta, R. (2015). \emph{Spatialfil: Application of 2D
convolution kernel filters to matrices or 3D arrays}. Retrieved from
\url{https://CRAN.R-project.org/package=spatialfil}

Douglas Nychka, Reinhard Furrer, John Paige, \& Stephan Sain. (2015).
Fields: Tools for spatial data. Boulder, CO, USA: University Corporation
for Atmospheric Research.
doi:\href{http://dx.doi.org/10.5065/D6W957CT}{10.5065/D6W957CT}

Edwards, W., Lindman, H., \& Savage, L. J. (1963). Bayesian statistical
inference for psychological research. \emph{Psychological Review},
\emph{70}, 193--242.

Eriksen, B. A., \& Eriksen, C. W. (1974). Effects of noise letters upon
the identification of a target letter in a nonsearch task.
\emph{Perception \& Psychophysics}, \emph{16}(1), 143--149.
doi:\href{http://dx.doi.org/10.3758/BF03203267}{10.3758/BF03203267}

Furrer, R., \& Sain, S. R. (2010). spam: A sparse matrix R package with
emphasis on MCMC methods for Gaussian Markov random fields.
\emph{Journal of Statistical Software}, \emph{36}(10), 1--25. Retrieved
from \url{http://www.jstatsoft.org/v36/i10/}

Gelfand, A., \& Smith, A. F. M. (1990). Sampling based approaches to
calculating marginal densities. \emph{Journal of the American
Statistical Association}, \emph{85}, 398--409.

Gelman, A., Carlin, J. B., Stern, H. S., \& Rubin, D. B. (2004).
\emph{Bayesian data analysis (2nd edition)}. London: Chapman; Hall.

Genz, A., \& Bretz, F. (2009). \emph{Computation of multivariate normal
and t probabilities}. Heidelberg: Springer-Verlag.

George, E. I., \& McCulloch, R. E. (1993). Variable selection via Gibbs
sampling. \emph{Journal of the American Statistical Association},
\emph{88}, 881--889.

Gerber, F., \& Furrer, R. (2015). Pitfalls in the implementation of
Bayesian hierarchical modeling of areal count data: An illustration
using BYM and Leroux models. \emph{Journal of Statistical Software, Code
Snippets}, \emph{63}(1), 1--32. Retrieved from
\url{http://www.jstatsoft.org/v63/c01/}

Goldstein, M. (2006). Subjective Bayesian analysis: Principles and
practice. \emph{Bayesian Analysis}, \emph{1}, 403--420. Retrieved from
\url{http://dx.doi.org/10.1214/06-ba116}

Greenwald, A., Klinger, M., \& Schuh, E. (1995). Activation by
marginally perceptible (``subliminal") stimuli: Dissociation of
unconscious from conscious cognition. \emph{Journal of Experimental
Psychology: General}, \emph{124}, 22--42.

Hays, W. L. (1994). \emph{Statistics} (fifth.). Ft. Worth, T.X.:
Harcourt Brace.

Jackman, S. (2009). \emph{Bayesian analysis for the social sciences}.
Chichester, United Kingdom: John Wiley \& Sons.

Jackson, C. H. (2011). Multi-state models for panel data: The msm
package for R. \emph{Journal of Statistical Software}, \emph{38}(8),
1--29. Retrieved from \url{http://www.jstatsoft.org/v38/i08/}

Jeffreys, H. (1961). \emph{Theory of probability (3rd edition)}. New
York: Oxford University Press.

Kahneman, D., \& Tversky, A. (1972). Subjective probability: A judgment
of representativeness. \emph{Cognitive Psychology}, \emph{3}, 430--454.
Retrieved from
\url{http://www.sciencedirect.com/science/article/pii/0010028572900163}

Kass, R. E., \& Raftery, A. E. (1995). Bayes factors. \emph{Journal of
the American Statistical Association}, \emph{90}, 773--795.

Klugkist, I., \& Hoijtink, H. (2007). The Bayes factor for inequality
and about equality constrained models. \emph{Computational Statistics \&
Data Analysis}, \emph{51}(12), 6367--6379.

Klugkist, I., Kato, B., \& Hoijtink, H. (2005). Bayesian model selection
using encompassing priors. \emph{Statistica Neerlandica}, \emph{59},
57--69.

Laplace, P. S. (1986). Memoir on the probability of the causes of
events. \emph{Statistical Science}, \emph{1}(3), 364--378. Retrieved
from \url{http://www.jstor.org/stable/2245476}

Lemon, J. (2006). Plotrix: A package in the red light district of r.
\emph{R-News}, \emph{6}(4), 8--12.

Liang, F., Paulo, R., Molina, G., Clyde, M. A., \& Berger, J. O. (2008).
Mixtures of g-priors for Bayesian variable selection. \emph{Journal of
the American Statistical Association}, \emph{103}, 410--423. Retrieved
from \url{http://pubs.amstat.org/doi/pdf/10.1198/016214507000001337}

Lu, C. H., \& Proctor, R. W. (1995). The influence of irrelevant
location information on performance: A review of the Simon and spatial
Stroop effects. \emph{Psychonomic Bulletin and Review}, \emph{2}(2),
174--207.

Luce, R. D. (1986). \emph{Response times}. New York: Oxford University
Press.

MacLeod, C. (1991). Half a century of research on the stroop effect: An
integrative review. \emph{Psychological Bulletin}, \emph{109}, 163--203.

Martin, A. D., Quinn, K. M., \& Park, J. H. (2011). MCMCpack: Markov
chain monte carlo in R. \emph{Journal of Statistical Software},
\emph{42}(9), 22. Retrieved from \url{http://www.jstatsoft.org/v42/i09/}

Meng, X.-L., \& Wong, W. H. (1996). Simulating ratios of normalizing
constants via a simple identity: A theoretical exploration.
\emph{Statistica Sinica}, 831--860.

Morey, R. D., \& Rouder, J. N. (2015). \emph{BayesFactor: Computation of
bayes factors for common designs}. Retrieved from
\url{https://CRAN.R-project.org/package=BayesFactor}

Morey, R. D., Romeijn, J.-W., \& Rouder, J. N. (2016). The philosophy of
Bayes factors and the quantification of statistical evidence.
\emph{Journal of Mathematical Psychology}, --. Retrieved from
\url{http://www.sciencedirect.com/science/article/pii/S0022249615000723}

Morey, R. D., Rouder, J. N., Pratte, M. S., \& Speckman, P. L. (2011).
Using MCMC chain outputs to efficiently estimate Bayes factors.
\emph{Journal of Mathematical Psychology}, \emph{55}, 368--378.
Retrieved from \url{http://dx.doi.org/10.1016/j.jmp.2011.06.004}

Ooms, J. (2016). \emph{Curl: A modern and flexible web client for r}.
Retrieved from \url{https://CRAN.R-project.org/package=curl}

Overstall, A. M., \& Forster, J. J. (2010). Default bayesian model
determination methods for generalised linear mixed models.
\emph{Computational Statistics \& Data Analysis}, \emph{54}(12),
3269--3288.

Plate, T., \& Heiberger, R. (2015). \emph{Abind: Combine
multidimensional arrays}. Retrieved from
\url{https://CRAN.R-project.org/package=abind}

Plummer, M., Best, N., Cowles, K., \& Vines, K. (2006). CODA:
Convergence diagnosis and output analysis for mCMC. \emph{R News},
\emph{6}(1), 7--11. Retrieved from
\url{http://CRAN.R-project.org/doc/Rnews/}

Pratte, M. S., Rouder, J. N., Morey, R. D., \& Feng, C. (2010).
Exploring the differences in distributional properties between Stroop
and Simon effects using delta plots. \emph{Attention, Perception \&
Psychophysics}, \emph{72}, 2013--2025.

R Core Team. (2016). \emph{R: A language and environment for statistical
computing}. Vienna, Austria: R Foundation for Statistical Computing.
Retrieved from \url{https://www.R-project.org/}

Raftery, A. E. (1996). Approximate Bayes factors and accounting for
model uncertainty in generalised linear models. \emph{Biometrika},
\emph{83}, 251--266.

Richard A. Becker, O. S. code by, Ray Brownrigg. Enhancements by Thomas
P Minka, A. R. W. R. version by, \& Deckmyn., A. (2016). \emph{Maps:
Draw geographical maps}. Retrieved from
\url{https://CRAN.R-project.org/package=maps}

Ridderinkhof, K. R. (1997). A dual-route processing architecture for
stimulus-response correspondence effects. In B. Hommel \& W. Prinz
(Eds.), \emph{Theoretical issues in S-R compatibility} (pp. 119--131).
Amsterdam, the Netherlands: Elsevier.

Robertson, T., Wright, F., \& Dykstra, R. (1988). \emph{Order restricted
statistical inference.} Wiley, New York.

Roitman, J. D., \& Shadlen, M. N. (2002). Response of neurons in the
lateral intraparietal area during a combined visual discrimination
reaction time task. \emph{Journal of Neuroscience}, \emph{22},
9475--9489.

Rouder, J. N., \& Lu, J. (2005). An introduction to Bayesian
hierarchical models with an application in the theory of signal
detection. \emph{Psychonomic Bulletin and Review}, \emph{12}, 573--604.

Rouder, J. N., \& Morey, R. D. (2012). Default Bayes factors for model
selection in regression. \emph{Multivariate Behavioral Research},
\emph{47}, 877--903. Retrieved from
\url{http://dx.doi.org/10.1080/00273171.2012.734737}

Rouder, J. N., Morey, R. D., \& Wagenmakers, E.-J. (2016). The interplay
between subjectivity, statistical practice, and psychological
scienceCollabra. \emph{Collabra}, \emph{2}, 6. Retrieved from
\url{http://doi.org/10.1525/collabra.28}

Rouder, J. N., Morey, R. D., Speckman, P. L., \& Province, J. M. (2012).
Default Bayes factors for ANOVA designs. \emph{Journal of Mathematical
Psychology}, \emph{56}, 356--374. Retrieved from
\url{http://dx.doi.org/10.1016/j.jmp.2012.08.001}

Rouder, J. N., Morey, R. D., Verhagan, A. J., Swagman, A., \&
Wagenmakers, E.-J. (2016). Bayesian analysis of factorial designs.
\emph{Psychological Methods}.

Rouder, J. N., Province, J. M., Morey, R. D., Gomez, P., \& Heathcote,
A. (2015). The lognormal race: A cognitive-process model of choice and
latency with desirable psychometric properties. \emph{Psychometrika}.

Rouder, J. N., Speckman, P. L., Sun, D., Morey, R. D., \& Iverson, G.
(2009). Bayesian \(t\)-tests for accepting and rejecting the null
hypothesis. \emph{Psychonomic Bulletin and Review}, \emph{16}, 225--237.
Retrieved from \url{http://dx.doi.org/10.3758/PBR.16.2.225}

Rouder, J. N., Yue, Y., Speckman, P. L., Pratte, M. S., \& Province, J.
M. (2010). Gradual growth vs. shape invariance in perceptual decision
making. \emph{Psychological Review}, \emph{117}, 1267--1274.

Simon, J. R. (1969). Reactions toward the source of stimulation.
\emph{Journal of Experimental Psychology}, \emph{81}, 174--176.

Speckman, P. L., Rouder, J. N., Morey, R. D., \& Pratte, M. S. (2008).
Delta plots and coherent distribution ordering. \emph{The American
Statistician}, \emph{62}, 262--266.

Stroop, J. R. (1935). Studies of interference in serial verbal
reactions. \emph{Journal of Experimental Psychology}, \emph{18},
643--662.

Swagman, A., Province, J., \& Rouder, J. (2015). Evidence for
discrete-state processing in perceptual word identification.
\emph{Psychonomic Bulletin \& Review}, \emph{22}, 265--273.

Sweldens, S., Corneille, O., \& Yzerbyt, V. (2014). The role of
awareness in attitude formation through evaluative conditioning.
\emph{Personality and Social Psychology Review}, \emph{18}(2), 187--209.

Venables, W. N., \& Ripley, B. D. (2002). \emph{Modern applied
statistics with s} (Fourth.). New York: Springer. Retrieved from
\url{http://www.stats.ox.ac.uk/pub/MASS4}

Von Bastian, C. C., Souza, A. S., \& Gade, M. (2015). No evidence for
bilingual cognitive advantages: A test of four hypotheses. \emph{Journal
of Experimental Psychology: General}, \emph{145}(2), 246--258.

Wagenmakers, E. J., \& Brown, S. (2007). On the linear relation between
the mean and the standard deviation of a response time distribution.
\emph{Psychological Review}, \emph{114}, 830--841.

Wagenmakers, E.-J., Lodewyckx, T., Kuriyal, H., \& Grasman, R. (2010).
Bayesian hypothesis testing for psychologists: A tutorial on the
Savage-Dickey method. \emph{Cognitive Psychology}, \emph{60}, 158--189.

West, R., Jakubek, K., Wymbs, N., Perry, M., \& Moore, K. (2005). Neural
correlates of conflict processing. \emph{Experimental Brain Research},
\emph{167}(1), 38--48.

Wetzels, R., \& Wagenmakers, E. (2012). A default Bayesian hypothesis
test for correlations and partial correlations. \emph{Psychonomic
Bulletin \& Review}, \emph{19}, 1057--1064.

Wetzels, R., Grasman, R. P., \& Wagenmakers, E.-J. (2010). An
encompassing prior generalization of the Savage-Dickey density ratio.
\emph{Computational Statistics and Data Analysis}, \emph{54},
2094--2102.

Wickham, H., \& Chang, W. (2016). \emph{Devtools: Tools to make
developing r packages easier}. Retrieved from
\url{https://CRAN.R-project.org/package=devtools}

Wilhelm, S., \& G, M. B. (2015). \emph{tmvtnorm: Truncated multivariate
normal and student t distribution}. Retrieved from
\url{http://CRAN.R-project.org/package=tmvtnorm}

Zeileis, A. (2004). Econometric computing with hC and hAC covariance
matrix estimators. \emph{Journal of Statistical Software},
\emph{11}(10), 1--17. Retrieved from
\url{http://www.jstatsoft.org/v11/i10/}

Zeileis, A. (2006). Object-oriented computation of sandwich estimators.
\emph{Journal of Statistical Software}, \emph{16}(9), 1--16. Retrieved
from \url{http://www.jstatsoft.org/v16/i09/.}

Zellner, A. (1986). On assessing prior distirbutions and Bayesian
regression analysis with \(g\)-prior distribution. In P. K. Goel \& A.
Zellner (Eds.), \emph{Bayesian inference and decision techniques: Essays
in honour of Bruno de Finetti} (pp. 233--243). Amsterdam: North Holland.

Zellner, A., \& Siow, A. (1980). Posterior odds ratios for selected
regression hypotheses. In J. M. Bernardo, M. H. DeGroot, D. V. Lindley,
\& A. F. M. Smith (Eds.), \emph{Bayesian statistics: Proceedings of the
First International Meeting held in Valencia (Spain)} (pp. 585--603).
University of Valencia.

Zhang, J., \& Kornblum, S. (1997). Distributional analyses and De Jong,
Liang and Lauber's (1994) dual-process model of the Simon Effect.
\emph{Journal of Experimental Psychology: Human Perception and
Performance}, \emph{23}, 1543--1551.



\end{document}
